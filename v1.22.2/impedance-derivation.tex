\documentclass[12pt]{amsproc}
\usepackage[a4paper]{geometry}
\usepackage{float}
\usepackage{fouriernc}
\usepackage{verbatim}

\begin{comment}
  Where to publish this ? JOSS is merely a software journal.
  - CPC ?
  - WIAS TR
  - As part of documenter ?
    After all, yes

    Computing in Science \& Engineering (CiSE)
    
cite:  https://www.researchgate.net/publication/223773028_A_frequency-domain_approach_to_dynamical_modeling_of_electrochemical_power_sources
https://www.sciencedirect.com/science/article/abs/pii/S0013468697002387
https://www.gamry.com/application-notes/EIS/basics-of-electrochemical-impedance-spectroscopy/
https://iopscience.iop.org/article/10.1149/09401.0251ecst


\end{comment}



\title{Small signal analysis for nonlinear evolution equations\\ Draft.}
\author{J. Fuhrmann}
\newcommand{\CV}{{\mathcal{V}}}
\newcommand{\VV}{\mathbb{V}}
\newcommand{\PP}{\mathbb{P}}
\newcommand{\TT}{\mathbb{T}}
\newcommand{\MM}{\mathbb{M}}
\newcommand{\RR}{\mathbb{R}}
\newtheorem{example}{Example}

\newcommand{\Exp}[1]{e^{#1}}
\begin{document}
\maketitle
%\begin{multicols}{2}
\section{Frequency response of an electrical network}
  
The  current  dependency  on  voltage   of  basic  basic  elements  of
electrical networks  can be subsummed  using the notion  of impedance.
Let $U(t)$  be a given  voltage difference  applied to as  device, and
$I(t)$ be the resulting current.  For a periodic voltage perturbation,
the  frequency dependent  impedance  describes the  ratio between  the
amplitudes  of the  voltagee and  the current  response, respectitvely.
Assuming a an applied  periodic voltage $U(t)=U_a\Exp{i\omega t}$ with
amplitude $U_a$ and frequency $\omega$, we get
\begin{table}[H]
  \begin{center}
\renewcommand{\arraystretch}{1.4}
\begin{tabular}{lllll}
Circuit Element     & Law     & Standard form & Complex Form & Impedance\\ 
Resistance  & Ohm     & $ I(t)=\frac1R U(t)$                                & $I(t)=\frac1R U(t)$       &  $Z(\omega)=R$ \\
Capacity    & Faraday & $ I(t)=C(t)\frac{d}{dt}U(t)$                        & $I(t)=Ci\omega U(t)$&          $Z(\omega)=\frac1{Ci\omega}$\\
Inductivity & Henry   & $ I(t)=I_0+\frac1L\int\limits_{t_0}^tU(\tau) d\tau$ & $I(t)=\frac1{Li\omega}$ U(t)&$Z(\omega)=Li\omega$ \\
\end{tabular}
\caption{\label{tab:impedance} Impedance of three basic elements of electric circuits}\hfill
\end{center}
\end{table}

Any  network consisting  of  these  elements can  be  analyzed in  the
complex domain  using Kirchoffs law  and regarding the impedance  as a
complex resistance.

\section{Impedance spectroscopy in nonlinear evolution equations}

Usually, the interpretations of impedance spectroscopy measurements of
other  systems,  e.g. electrochemical  systems  is  performed using  a
replacement circuit consisting of  electrical network elements.  Being
quite  successful  in many  cases,  this  interpretation basically  is
limited to compartment type models.

Here,  we discuss an  approach of  applying impedance  spectroscopy to
abstract evolution equations based on small signal analysis.

For a  given time  interval $\TT=[0,T]$, a  Banach space  $\CV$ called
{\em state space},  and a finite dimensional space  $\PP$, called {\em
parameter  space},  regard  the  abstract doubly  nonlinear  evolution
equation
\begin{equation}\label{eq:abstrevol}
 \frac{d S(v(t),\lambda)}{dt} + D(v(t),\lambda)=0
\end{equation}

The state of  the system is {\em measured} by  some functional $M: \CV
\rightarrow  \MM$,   where  $\MM$  is  the   finite  dimensional  {\em
measurement space}. We assume  $M(v)=M^{stdy}(v)+ d_t M^{tran}(v)$.

As an example, $\CV$ may  be a finite dimensional space containing the
solution of some discretized system of partial differential equations.


Given   a  steady   state  $(v_0,   \lambda_0)$  such   that  $D(v_0,
\lambda_0)=0$, measured  by $M_0=M(v_0)$, we  would like to  trace its
response  to  a small,  periodic  perturbation $\lambda(t)=  \lambda_a
\Exp{i\omega  t}   $.   Expressing   the measurement of this  response  as   $M(t)=  M_0+
M_a(\omega)  \Exp{i\omega  t}$,  we  yield the  impedance  $Z(\omega)=
M_a(\omega)^{-1} \lambda_a$.



In order to calculate the frequency response, we make the ansatz $v(t)=v_0+v_a\Exp{i\omega  t}$ for
the perturbation of the state variable and calculate the first order Taylor expansions
of the terms $S,D$:
\begin{equation*}
  \begin{split}
    S(v_0+v_a\Exp{i\omega  t},\lambda_0+\lambda_a\Exp{i\omega  t})\approx&S(v_0,\lambda_0)+ 
          S_v(v_0,\lambda_0)v_a\Exp{i\omega  t}+
          S_\lambda(v_0,\lambda_0)\lambda_a\Exp{i\omega  t}\\
    D(v_0+v_a\Exp{i\omega  t},\lambda_0+\lambda_a\Exp{i\omega  t})\approx&D(v_0,\lambda_0)+ 
          D_v(v_0,\lambda_0)v_a\Exp{i\omega  t}+
          D_\lambda(v_0,\lambda_0)\lambda_a\Exp{i\omega  t}
  \end{split}
\end{equation*}

Putting them into equation \eqref{eq:abstrevol}  and using the steady state condition yields
\begin{equation*}
  \begin{split}
    \frac{d}{dt}\left( 
      S_v(v_0,\lambda_0)v_a\Exp{i\omega  t}+
      S_\lambda(v_0,\lambda_0)\lambda_a\Exp{i\omega  t}\right)+ 
    D_v(v_0,\lambda_0)v_a\Exp{i\omega  t}+
    D_\lambda(v_0,\lambda_0)\lambda_a\Exp{i\omega  t}&=0\\
    i\omega\left( 
      S_v(v_0,\lambda_0)v_a\Exp{i\omega  t}+
      S_\lambda(v_0,\lambda_0)\lambda_a\Exp{i\omega  t}\right)+ 
    D_v(v_0,\lambda_0)v_a\Exp{i\omega  t}+
    D_\lambda(v_0,\lambda_0)\lambda_a\Exp{i\omega  t}&=0\\
    i\omega\left( 
      S_v(v_0,\lambda_0)v_a+
      S_\lambda(v_0,\lambda_0)\lambda_a\right)+ 
    D_v(v_0,\lambda_0)v_a+
    D_\lambda(v_0,\lambda_0)\lambda_a&=0
  \end{split}
\end{equation*}


For the measurement, we have
\begin{align*}
      M(v_0+v_a\Exp{i\omega  t})\approx&M^{stdy}(v_0)+ 
    M^{stdy}_v(v_0)v_a\Exp{i\omega  t} +  i\omega M^{tran}_v(v_0)v_a\Exp{i\omega  t}
\end{align*}

Assuming $S_\lambda=0$, $\dim \MM =\dim \PP=1$ we arrive at solving 
\begin{equation}\label{eq:impedsolve0}
    i\omega 
      S_v(v_0)v_a+
    D_v(v_0,\lambda_0)v_a+
    \lambda_a D_\lambda(v_0,\lambda_0)=0
\end{equation}
for given $\omega$ for the  unknown $v_a$.
The corresponding measurement yields:
\begin{align*}
  M_a(\omega)= M^{stdy}_v(v_0)v_a + i\omega M^{tran}_v(v_0)v_a 
\end{align*}

The impedance then can be calculated as 
\begin{equation*}\label{eq:imped0}
  Z(\omega)= \frac{\lambda_a}{M_a(\omega)}
\end{equation*}

Dividing \eqref{eq:impedsolve0} by  $\lambda_a$ (and thus rescaling $v_a$) gives the final expressions
\begin{equation}\label{eq:impedsolve}
    i\omega 
      S_v(v_0)v_a+
    D_v(v_0,\lambda_0)v_a+
    D_\lambda(v_0,\lambda_0)=0
\end{equation}
and
\begin{equation}\label{eq:imped}
  Z(\omega)= \frac{1}{ M^{stdy}_v(v_0)v_a + i\omega M^{tran}_v(v_0)v_a}.
\end{equation}

\subsection{Excited Dirichlet boundary condition}
This dicussion is based on the idea to implement Dirichlet boundary conditions using the
penalty method and provides an easy way to handle deriviatives with respect to
the Dirichlet boundary value. We assume that our problem corresponds to a (discretized) PDE
in a domain $\Omega$ with boundary $\Gamma=\partial\Omega$.
\begin{example}{Dirichlet Boundary Conditions}\\
  Assume that $\Gamma_0,\Gamma_1\subset \Gamma$  are two disjunct parts of
  the boundary.
  For (discretized) differential operator $D_i$ defined
  in $\Omega$ assume that $D_i(v)=0$ is the discretization of the
  homogeneous Neumann boundary value problem. Adding Robin boundary value terms
  for $\Gamma_0, \Gamma_1$ with Robin coefficient $\frac1\epsilon$ and boundary values
  0, resp. $\lambda$ can be formally
  written as
  \begin{align*}
  D(v,\lambda)= D_i(v)
  +\frac1\epsilon\delta_{\Gamma_0}(v)
  +\frac1\epsilon\delta_{\Gamma_1}(v-\lambda),    
  \end{align*}
  where $\delta$ means delta functions with support at the corresponding boundary.
  The limit $\epsilon\to 0$  corresponds to the Dirichlet boundary condition.
  This covers the case of an
  applied voltage difference  and an excitation at $\Gamma_1$.
  

  Then  
\begin{equation*}
  \begin{split}
    D_v(v,\lambda)= D_{i,v}(v)+ \frac1\epsilon\delta_{\Gamma_0}+ \frac1\epsilon\delta_{\Gamma_1}\\
    D_\lambda(v,\lambda)= -\frac1\epsilon\delta_{\Gamma_1}\\
   \end{split}
\end{equation*}
Therefore  for the impedance calculation we have to solve 
\begin{equation*}
  i\omega S_v v_a + D_{i,v} v_a +\frac1\epsilon\delta_{\Gamma_0}v_a +\frac1\epsilon\delta_{\Gamma_1}(v_a-1) =0
\end{equation*}
which for $\epsilon\to 0$ corresponds to the Dirichlet problem  for the linearized equation 
with boundary condition 0 on $\Gamma_0$ and 1 on $\Gamma_1$.
\end{example}

\subsection{Flux calculation via test function}
For a discussion, see e.g. [Gajewski, WIAS TR],  [Yoder, Gärtner], [Farrell et al, Handbook].

Assume that $D_i(v,\lambda) = \nabla \cdot J(v) + R(v)$. Let $T$ be be a test function
such that $\Delta T    +\frac1\epsilon\delta_{\Gamma_0}(T)
+\frac1\epsilon\delta_{\Gamma_1}(T-1)=0$, which for $\epsilon\to 0$  corresponds to
a mixed boundary value problem with $T=0$ on $\Gamma_0$, $T=1$ on $\Gamma_1$ and $\partial_n T=0$ on
$\partial\Omega \setminus  (\Gamma_0 \cup \Gamma_1)$.
Then the boundary flux integral as measurement can be calculated as
\begin{align*}
  M(v)&=\int_{\Gamma_1} J(v,\lambda)\cdot n ds
      =\int_{\Gamma_1} T J(v,\lambda)\cdot n ds
      =\int_{\Gamma} T J(v,\lambda)\cdot n ds\\
      &=\int_{\Omega} \nabla\cdot \left(T J(v,\lambda)\right) d\vec x
      =\int_\Omega \left(J(v) \nabla T + T (\nabla \cdot J)\right) d\vec x\\
      &=\int_\Omega \left(J(v) \nabla T + T (R(v,\lambda) + d_t S(v))\right) d\vec x
\end{align*}


Accordingly, $M^{stdy}(v)=\int_\Omega \left(J(v) \nabla T + T R(v)\right) d\vec x$ and
  $M^{tran}(v) = \int_{\Omega} S(v) T d\vec x$
and
$$M_a(\omega)= \int_\Omega \left(J_v(v_0)\nabla v_a \nabla T + T (R_v(v_0)v_a + i\omega S_v(v_0)v_a)\right) d\vec x$$

With a finite volume discretizatiom, this representation has a discrete counterpart.


\subsection{Examples with analytical impedance expressions}
These examples can be taken for benchmarking numerical methods.
\begin{example}{Current Response}
  We calculate the impedance of the current response at $L$ to voltage change in $0$ of the linear reaction  diffusion system in $(0,L)$  
  \begin{equation*}
    \begin{cases}
      Cu_t - (Du_x)_x + Ru=&0\\
      u(0,t)=&\lambda\\
      u(L,t)=&0\\
    \end{cases}
  \end{equation*}
    As response functions we take the currents  $I_0=Du_x(0,t)$ and $I_L=-Du_x(L,t)$.
    The corresponding impedance equation is
    \begin{equation*}
      \begin{cases}
        Ci\omega v - (Dv_x)_x +Rv =&0\\
        v(0,t)=&1\\
        v(L,t)=&0.\\
      \end{cases}
    \end{equation*}
Setting $z=\sqrt{i\omega\frac{C}{D}+\frac{R}{D}}$, for the solution, we make the ansatz
\begin{equation*}
  v=ae^{zx}+be^{-zx}
\end{equation*}
which fulfills the differential equation.
Setting $e^+=e^{zL},e^-=e^{-zL}$, from the boundary 
conditions we get the system
\begin{equation*}
  \begin{cases}
    a+b&=1\\
    ae^++be^-&=0\\
  \end{cases}
\end{equation*}
with the solutions $a=\frac{e^-}{e^--e^+},b=\frac{e^+}{e^+-e^-}$
Therefore, we have
\begin{equation*}
  Dv_x(0)=Dz(a-b)=Dz\frac{e^-+e^+}{e^--e^+}
\end{equation*}
and
\begin{equation*}
  Dv_x(L)=Dz(ae^+-be^-)=Dz\frac{e^-e^++e^+e^-}{e^--e^+}=\frac{2Dz}{e^--e^+}
\end{equation*}

Alternatively, we might be interested in e.g. $M(v)=\int_\Omega |\nabla v|^2$
as measurement. We have
\begin{align*}
  \nabla v&=a z e^{zx} - b z e^{-zx}\\
  |\nabla v|^2&=a z e^{2zx} -ab +  b z e^{-2zx}\\
  \int  |\nabla v|^2 \; dx &=  \frac{1}{2z}aze^{2zx}  -\frac{1}{2z}  b z e^{-2zx} - abx + C\\
          &= \frac{a}{2}e^{2zx}  -\frac{b}{2}  b e^{-2zx} -abx +C\\
  \int_0^L  |\nabla v|^2 \; dx &=   \frac{a}{2}e^{2zL}  -\frac{b}{2}  e^{-2zL} -abL
            -  \frac{a}{2}  +\frac{b}{2}\\
  \int_0^L  |\nabla v|^2 \; dx &=   \frac{a}{2}(e^{2zL}-1)  -\frac{b}{2} (e^{-2zL}-1) -abL\\
                               &=    \frac12\frac{e^{-zL}}{e^{-zL}-e^{zL}}(e^{2zL}-1)  -\frac12\frac{e^{zL}}{e^{zL}-e^{-zL}} (e^{-2zL}-1) -abL\\
          &=    \frac12\frac{e^{zL}-e^{-zL}}{e^{-zL}-e^{zL}}  -\frac12\frac{e^{-zL}-e^{zL}}{e^{zL}-e^{-zL}} -abL\\
              &=-abL\\
  &= \frac{L}{e^{2zL} + e^{-2zL}-2}
\end{align*}

But ... the measurement is different: it has to be the derivative at $v_0$ applied to the solution:
\begin{align*}
  2\int_\omega \nabla v_0 \nabla v_a 
\end{align*}
Steady state: $u(0)=1$, $u(L)=0$. Let $d=\sqrt{\frac{R}{D}}$.
Set $u(x)=fe^{dx}+ ge^{-dx}$. 
\begin{align*}
  u'(x) &= fd e^{dx} -gde^{-dx}\\
  u''(x) &= fd^2 e^{dx} +gd^2e^{-dx}=d^2 u(x)\\
  u(0)=f+g\\
  u(L)=fe^{dL}+ge^{dL}
\end{align*}
So for $d^+=e^{dL}$ and $d^-=e^{-dL}$ we get
$f=\frac{d^-}{d^--d^+}$ and $g=\frac{d^+}{d^+-d^-}$ 

So we have to calculate
\begin{align*}
  \nabla v_0 \nabla v_a&=  (aze^{zx}-bze^{-zx})(fde^{dx}-gde^{-dx})\\
                       &= azfde^{(z+d)x} -azgde^{(z-d)x}  - bzfde^{(d-z)x} + bzgde^{-(z+d)x}\\
  \int \nabla v_0 \nabla v_a &=   \frac{azfd}{z+d}e^{(z+d)x} -\frac{azgd}{z-d}e^{(z-d)x}  - \frac{bzfd}{d-z}e^{(d-z)x} - \frac{bzgd}{z+d}e^{-(z+d)x}\\
  &=  zd\left(\frac{af}{z+d}e^{(z+d)x} -\frac{ag}{z-d}e^{(z-d)x}  - \frac{bf}{d-z}e^{(d-z)x} - \frac{bg}{z+d}e^{-(z+d)x}\right)
\end{align*}

\end{example}


\begin{example}{Voltage response}
Regard the system
  \begin{equation*}
    \begin{cases}
      Cu_t - (Du_x)_x + Ru=&0\\
      u(0,t)=&\lambda\\
      (Du_x)(L,t)=&0\\
    \end{cases}
  \end{equation*}
  and observe the values $u(L)$ and $Du_x(0)$.
The same ansatz as in the previous example leads to 
\begin{equation*}
  \begin{cases}
    a+b&=1\\
    izae^+-izbe^-&=0\\
  \end{cases}
\end{equation*}
with the solutions $a=\frac{e^-}{e^++e^-}$ and $b=\frac{e^+}{e^++e^-}$
Therefore,  $Du_x(0)=iDz\frac{e^--e^+}{e^++e^-}$ and
$u(L)=\frac2{e^++e^-}$.
\end{example}


\section{Algorithmic implementation}

Algorithmically, we proceed as follows:

\begin{description}
\item[1] Given $\lambda$, solve steady state equation $D(v,\lambda)=0$
\item[2] Obtain Jacobi matrix for stationary problem
\item[3] Obtain Jacobi matrix of time derivative
\item[4] Obtain Derivative of measurement (both steady state and transient parts)
\item[5] Prepare system for impedance problem
\item[6] For $\omega=\omega_0\dots\omega_1$:
  \begin{description}
  \item[6.1] set up complex matrix
  \item[6.2] solve complex system, take old solution as initial value if this is done iteratively
  \item[6.3]calculate impedance functional
  \end{description}
\end{description}

\end{document}










